%-------------------------
% Resume in Latex
% Author : Sayan Nath
% Based off of: https://github.com/sayannath/Resume-LaTex
% License : MIT
%------------------------

\documentclass[letterpaper,11pt]{article}

\usepackage{latexsym}
\usepackage[empty]{fullpage}
\usepackage{titlesec}
\usepackage{marvosym}
\usepackage[usenames,dvipsnames]{color}
\usepackage{verbatim}
\usepackage{enumitem}
\usepackage[hidelinks]{hyperref}
\usepackage{fancyhdr}
\usepackage[english]{babel}
\usepackage{tabularx}
\input{glyphtounicode}


%----------FONT OPTIONS----------
% sans-serif
% \usepackage[sfdefault]{FiraSans}
\usepackage[sfdefault]{roboto}
% \usepackage[sfdefault]{noto-sans}
% \usepackage[default]{sourcesanspro}

% serif
% \usepackage{CormorantGaramond}
% \usepackage{charter}


\pagestyle{fancy}
\fancyhf{} % clear all header and footer fields
\fancyfoot{}
\renewcommand{\headrulewidth}{0pt}
\renewcommand{\footrulewidth}{0pt}

% Adjust margins
\addtolength{\oddsidemargin}{-0.5in}
\addtolength{\evensidemargin}{-0.5in}
\addtolength{\textwidth}{1in}
\addtolength{\topmargin}{-.5in}
\addtolength{\textheight}{1.0in}

\urlstyle{same}

\raggedbottom
\raggedright
\setlength{\tabcolsep}{0in}

% Sections formatting
\titleformat{\section}{
  \vspace{-4pt}\scshape\raggedright\large
}{}{0em}{}[\color{black}\titlerule \vspace{-5pt}]

% Ensure that generate pdf is machine readable/ATS parsable
\pdfgentounicode=1

%-------------------------
% Custom commands
\newcommand{\resumeItem}[1]{
  \item\small{
    {#1 \vspace{-2pt}}
  }
}

\newcommand{\resumeSubheading}[4]{
  \vspace{-2pt}\item
    \begin{tabular*}{0.97\textwidth}[t]{l@{\extracolsep{\fill}}r}
      \textbf{#1} & #2 \\
      \textit{\small#3} & \textit{\small #4} \\
    \end{tabular*}\vspace{-7pt}
}

\newcommand{\resumeSubSubheading}[2]{
    \item
    \begin{tabular*}{0.97\textwidth}{l@{\extracolsep{\fill}}r}
      \textit{\small#1} & \textit{\small #2} \\
    \end{tabular*}\vspace{-7pt}
}

\newcommand{\resumeProjectHeading}[2]{
    \item
    \begin{tabular*}{0.97\textwidth}{l@{\extracolsep{\fill}}r}
      \small#1 & #2 \\
    \end{tabular*}\vspace{-7pt}
}

\newcommand{\resumeSubItem}[1]{\resumeItem{#1}\vspace{-4pt}}

\renewcommand\labelitemii{$\vcenter{\hbox{\tiny$\bullet$}}$}

\newcommand{\resumeSubHeadingListStart}{\begin{itemize}[leftmargin=0.15in, label={}]}
\newcommand{\resumeSubHeadingListEnd}{\end{itemize}}
\newcommand{\resumeItemListStart}{\begin{itemize}}
\newcommand{\resumeItemListEnd}{\end{itemize}\vspace{-5pt}}

%-------------------------------------------
%%%%%%  RESUME STARTS HERE  %%%%%%%%%%%%%%%%%%%%%%%%%%%%


\begin{document}

%----------HEADING----------
\begin{center}
    \textbf{\Huge \scshape Sayan Nath} \\ \vspace{1pt}
    \small +91 9874948947 $|$ \href{mailto:sayannath235@gmail.com}{\underline{sayannath235@gmail.com}} $|$ 
    \href{https://www.linkedin.com/in/sayannath235/}{\underline{linkedin.com/in/sayannath235}} $|$
    \href{https://github.com/sayannath}{\underline{github.com/sayannath}}
\end{center}


%-----------EDUCATION-----------
\section{Education}
  \resumeSubHeadingListStart
    \resumeSubheading
      {Kalinga Institute of Industrial Technology}{Bhubaneswar, India}
      {Bachelor of Technology in Information Technology, CGPA-9.02/10}{July. 2019 -- Present}
    % \resumeSubheading
    %   {Blinn College}{Bryan, TX}
    %   {Associate's in Liberal Arts}{Aug. 2014 -- May 2018}
  \resumeSubHeadingListEnd


%-----------EXPERIENCE-----------
\section{Experience}
  \resumeSubHeadingListStart
    \resumeSubheading
      {Google Summer of Code 2021, Student}{May 2021 -- Present}
      {TensorFlow}{Remote}
      \resumeItemListStart
        \resumeItem{Update the existing \href{https://github.com/tensorflow/examples/tree/master/lite/examples/bert_qa}{\textbf{BERT QnA}}, \href{https://github.com/tensorflow/examples/tree/master/lite/examples/text_classification}{\textbf{Text Classification}} and \href{https://github.com/tensorflow/examples/tree/master/lite/examples/smart_reply/android}{\textbf{Smart Reply}} examples in Android (Java) and iOS (Swift)}
        \resumeItem{Build and test the example in \textbf{Android (Java)} and \textbf{iOS (Swift)}. \textbf{Modify} this example to use the \href{https://www.tensorflow.org/lite/inference_with_metadata/task_library/overview}{\textbf{TFLite Task Library}} instead.}
      \resumeItemListEnd
      
    \resumeSubheading
      {Data Science Intern}{April 2021 -- Present}
      {Juppiter AI Labs}{Pune, IND}
      \resumeItemListStart
        \resumeItem{Worked with \textbf{OCR model} for converting Image to Text with \textbf{Keras}. Used that OCR model in \textbf{Flutter App}}
        \resumeItem{Created the backend in \textbf{NodeJs}, \textbf{Express}, \textbf{Redis}, \textbf{Socket.io}, and and maintained \textbf{AWS server} with \textbf{EC2} and \textbf{S3} instances.}
        \resumeItem{Scaled complete backend with \textbf{horizontal scaling} to increase efficiency by \textbf{70\%}}
    \resumeItemListEnd

    \resumeSubheading
      {Machine Learning Engineer}{Feb. 2021 -- Present}
      {Codebugged AI}{Delhi, IND}
      \resumeItemListStart
        \resumeItem{\textbf{Played} a major role for developing \textbf{Ikshana AI}. Worked with \textbf{TensorFlow Object Detection API} to \textbf{Computer-Vision} Models and used \textbf{TensorFlow JS} and \textbf{ReactJS} to deploy them in Web.}
        \resumeItem{Identified \textbf{Software} \& \textbf{DB architecture} based on project \textbf{requirements}, performing version control \&, refining product.}
        \resumeItem{Made many \textbf{Computer-Vision} models as per customer requirements. \textbf{Optimised} the models with various optimisation techniques like \textbf{Quantisation}, \textbf{Pruning} and \textbf{TensorRT}.}
        \resumeItem{Worked with \textbf{TF-Lite} to make inference in edge devices like Android and iOS Apps.}
      \resumeItemListEnd

  \resumeSubHeadingListEnd


%-----------PROJECTS-----------
\section{Projects}
    \resumeSubHeadingListStart
      \resumeProjectHeading
          {\textbf{Sanus \- A CADx Platform } $|$ \emph{TensorFlow, Flask, MERN, Flutter \& Docker} $|$ \href{https://github.com/SANUS-ML/SANUS-WEB}{GitHub}}{January 2021 -- Present}
          \resumeItemListStart
            \resumeItem{\textbf{Developed} a full-stack web application using with \textbf{Flask} and \textbf{NodeJs} serving a \textbf{REST API} with \textbf{ReactJS} as the frontend and \textbf{Flutter} as mobile application.}
            \resumeItem{The application can \textbf{detect diseases} with \textbf{one click} by upload the patient \textbf{electronic healthcare records} and \textbf{provide information} of various anomalies.}
            \resumeItem{In our platform, we are tackling \textbf{five major diseases} for now. We have deployed \textbf{light weighted} \textbf{Machine Learning} and \textbf{Deep Learning Models}.}
          \resumeItemListEnd
      \resumeProjectHeading
          {\textbf{MIRNet-Flutter} $|$ \emph{Flutter, TF-Lite \& Git} $|$ \href{https://github.com/sayannath/MIRNet-Flutter}{GitHub}}{May 2018 -- May 2020}
          \resumeItemListStart
            \resumeItem{It is used to enhance a \textbf{low-light image} to a \textbf{bright image}.}
            \resumeItem{This project got a mention in \textbf{Google I/O 2021} as a notable \textbf{Open Source Project}. It is also enlisted in the \textbf{ML-GDE}'s \href{https://github.com/ml-gde/e2e-tflite-tutorials}{\textbf{E2E TFLite Tutorials}} repository.}
          \resumeItemListEnd
      \resumeProjectHeading
          {\textbf{American-Sign-Language-Detection} $|$ \emph{Android,TensorFlow TF-Lite \& Git} $|$ \href{https://github.com/sayannath/American-Sign-Language-Detection}{GitHub}}{Feb. 2021 -- April 2021}
          \resumeItemListStart
            \resumeItem{It is a \textbf{deep learning} based app where we can detect \textbf{American Sign Language}}
            \resumeItem{Used \textbf{MobileNetV2 architecture} to train the images. It handles up to \textbf{29 classes}.}
            \resumeItem{Used \textbf{Quantization Aware Training} to \textbf{optimise} the model and exported it to \textbf{TF-Lite} for deploying it in android app.}
            \resumeItem{The \textbf{average inference time} is \textbf{172ms}}
          \resumeItemListEnd
    \resumeSubHeadingListEnd

%
%-----------PUBLICATIONS-----------
\section{Publications}
  \resumeSubHeadingListStart
    \resumeSubheading
      {Rahat - A Disaster Management App}{}
      {In Project Innovations in Distributed Computing and Internet Technology, 17th ICDCIT}{}
      \resumeItemListStart
      \resumeItem{\href{https://drive.google.com/file/d/16IL_54gRIlrMemAvtV0bdY24hkztj0pV/view}{Link}}
      \resumeItemListEnd
  \resumeSubHeadingListEnd

%
%-----------PROGRAMMING SKILLS-----------
% \section{Technical Skills}
%  \begin{itemize}[leftmargin=0.15in, label={}]
%     \small{\item{
%      \textbf{Languages}{: Java, Python, JavaScript, Dart} \\
%      \textbf{Frameworks}{: TensorFlow, Keras, NodeJS, Flask, Flutter} \\
%      \textbf{Databases}{: MongoDB, PostgreSQL, MySQL, Redis} \\
%      \textbf{Developer Tools}{: Git, Docker, Linux, Digital-Ocean, Google Cloud Platform, AWS}
%     }}
%  \end{itemize}


%-------------------------------------------
\end{document}

